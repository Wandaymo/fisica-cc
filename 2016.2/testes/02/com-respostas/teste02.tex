\documentclass[12pt,a4paper,oneside]{article}

\usepackage[utf8]{inputenc}
\usepackage[portuguese]{babel}
\usepackage[T1]{fontenc}
\usepackage{amsmath}
\usepackage{amsfonts}
\usepackage{amssymb}
\usepackage{graphicx}

\usepackage{xcolor}
% Definindo novas cores
\definecolor{verde}{rgb}{0.25,0.5,0.35}
\definecolor{jpurple}{rgb}{0.5,0,0.35}
% Configurando layout para mostrar codigos Java
\usepackage{listings}
\definecolor{mygreen}{rgb}{0,0.6,0}
\definecolor{mygray}{rgb}{0.5,0.5,0.5}
\definecolor{mymauve}{rgb}{0.58,0,0.82}

\usepackage{listings}
\lstdefinelanguage{JavaScript}{
	keywords={typeof, new, true, false, catch, function, return, null, catch, switch, var, if, in, while, do, else, case, break},
	keywordstyle=\color{blue}\bfseries,
	ndkeywords={class, export, boolean, throw, implements, import, this},
	ndkeywordstyle=\color{darkgray}\bfseries,
	identifierstyle=\color{black},
	sensitive=false,
	comment=[l]{//},
	morecomment=[s]{/*}{*/},
	commentstyle=\color{purple}\ttfamily,
	stringstyle=\color{red}\ttfamily,
	morestring=[b]',
	morestring=[b]",
}

\lstset{ %
	backgroundcolor=\color{white},   % choose the background color; you must add \usepackage{color} or \usepackage{xcolor}
	basicstyle=\small,        % the size of the fonts that are used for the code
	breakatwhitespace=false,         % sets if automatic breaks should only happen at whitespace
	breaklines=true,                 % sets automatic line breaking
	captionpos=b,                    % sets the caption-position to bottom
	commentstyle=\color{mygreen},    % comment style
	deletekeywords={...},            % if you want to delete keywords from the given language
	escapeinside={\%*}{*)},          % if you want to add LaTeX within your code
	extendedchars=true,              % lets you use non-ASCII characters; for 8-bits encodings only, does not work with UTF-8
	frame=single,	                   % adds a frame around the code
	keepspaces=true,                 % keeps spaces in text, useful for keeping indentation of code (possibly needs columns=flexible)
	keywordstyle=\color{blue},       % keyword style
	language=HTML,                 % the language of the code
	otherkeywords={*,...},           % if you want to add more keywords to the set
	numbers=left,                    % where to put the line-numbers; possible values are (none, left, right)
	numbersep=5pt,                   % how far the line-numbers are from the code
	numberstyle=\tiny\color{mygray}, % the style that is used for the line-numbers
	rulecolor=\color{black},         % if not set, the frame-color may be changed on line-breaks within not-black text (e.g. comments (green here))
	showspaces=false,                % show spaces everywhere adding particular underscores; it overrides 'showstringspaces'
	showstringspaces=false,          % underline spaces within strings only
	showtabs=false,                  % show tabs within strings adding particular underscores
	stepnumber=1,                    % the step between two line-numbers. If it's 1, each line will be numbered
	stringstyle=\color{mymauve},     % string literal style
	tabsize=2,	                   % sets default tabsize to 2 spaces
	title=\lstname,                   % show the filename of files included with \lstinputlisting; also try caption instead of title
	moredelim=**[is][\color{purple}]{@}{@},
}


\author{\\Universidade Federal de Goiás (UFG) - Regional Jataí\\Bacharelado em Ciência da Computação \\Física para Ciência da Computação \\Esdras Lins Bispo Jr.}

\title{\sc \huge Segundo Teste}

\date{26 de janeiro de 2016}

\begin{document}

\maketitle

{\bf ORIENTAÇÕES PARA A RESOLUÇÃO}

\footnotesize

\begin{itemize}
	\item A avaliação é individual, sem consulta;
	\item A pontuação máxima desta avaliação é 10,0 (dez) pontos, sendo uma das 05 (cinco) componentes que formarão a média final da disciplina: dois testes, duas provas e exercícios-bônus;
	\item A média final ($MF$) será calculada assim como se segue
	\begin{eqnarray}
		MF & = & MIN(10, S) \nonumber \\
		S & = & (\sum_{i=1}^{4} 0,2.T_i ) + 0,2.P  + EB \nonumber
	\end{eqnarray}
	em que 
	\begin{itemize}
		\item $S$ é o somatório da pontuação de todas as avaliações,
		\item $T_i$ é a pontuação obtida no teste $i$,
		\item $P$ é a pontuação obtida na prova, e
		\item $EB$ é a pontuação total dos exercícios-bônus.
	\end{itemize}
	\item O conteúdo exigido compreende os seguintes pontos apresentados no Plano de Ensino da disciplina: (2) Medidas Físicas e Vetores, e (3) Movimentos.
\end{itemize}


\begin{center}
	\fbox{\large Nome: \hspace{10cm}}
	\fbox{\large Assinatura: \hspace{9cm}}
\end{center}

\newpage

\normalsize

\begin{enumerate}

	\item (5,0 pt) {\bf (Halliday 2.15)} Se a posição de uma partícula é dada por $x = 4 -12t  + 3t^2$ (onde $t$ está em segundos e $x$ em metros):
		\begin{enumerate}
			\item Qual é a velocidade da partícula em $t = 1$s? 
			\item O movimento nesse instante é no sentido positivo ou negativo de $x$? 
			\item Qual é a velocidade escalar da partícula nesse instante?
			\item A velocidade escalar está aumentando	ou diminuindo nesse instante?
			\item Existe algum instante no 	qual a velocidade se anula? Caso a resposta seja afirmativa, para que valor de $t$ isso acontece? 
			\item Existe algum instante após  $t  = 3$s	no qual a partícula está se movendo no sentido negativo de $x$?  Caso a resposta seja afirmativa, para que valor de $t$ isso acontece?
		\end{enumerate}
	
	\vspace{0.3cm}
	
	{ \color{blue} 
		{\bf Resposta:} 
			\begin{enumerate}
				\item $x = 4 -12t  + 3t^2$ \ \ 
					$\therefore$ \ \  $v = -12 + 6t$ (derivada primeira)\\
					Para $t = 1$ s, temos \\
					$v = -12 + 6.1 = -6$ m/s
				\item Sentido negativo, pois a velocidade tem valor negativo.
				\item $v_{esc} = |v| = 6$ m/s
				\item Está diminuindo. Pois a aceleração do objeto é positiva \\
				$v = -12 + 6t$ \ \ 
				$\therefore$ \ \  $a = 6$ (derivada primeira)\\
				e a velocidade instantânea é negativa. Logo, a tendência é a velocidade escalar diminuir (e não o oposto).
				\item Sim, quando $t = 2$ s. \\
					  Para $v=0$, temos\\
					  $0 = -12 + 6t$ \ \ $\therefore$ \ \ $6t = 12$ \ \ $\therefore$ \ \ $t = 2$ s.
				\item Não, não existe. Pois para qualquer valor de $t >  2$ s, a velocidade instantânea será sempre positiva.
			\end{enumerate}
	}
	
	\newpage
	
	\item Em JavaScript, crie uma função {\tt velocidadeEscalarMedia} que receba quatro parâmetros: (i) {\tt x1} (posição inicial), (ii) {\tt x2} (posição final), (iii) {\tt t1} (instante inicial), e (iv) {\tt t2} (instante final). A função deve retornar um número (a velocidade escalar média). É necessário validar a entrada para garantir que  {\tt t2 - t1} seja positivo e não nulo. Se a entrada não for válida, a função deve imprimir, via {\tt console.log}, uma mensagem de erro.
	
	{\color{blue} \bf Resposta: }
	
\begin{lstlisting}[language=JavaScript]
function velocidadeEscalarMedia(x1, x2, t1, t2){
	if(t2 - t1 <= 0){
		console.log("Variacao de tempo invalida!");
	}
	else{
		var vm = (x2 - x1)/(t2 - t1);
		vm = Math.abs(vm);
		return vm;
	}
}\end{lstlisting}
	
	\end{enumerate}
\end{document}