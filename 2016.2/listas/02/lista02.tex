\documentclass[12pt,a4paper,oneside]{article}

\usepackage[utf8]{inputenc}
\usepackage[portuguese]{babel}
\usepackage[T1]{fontenc}
\usepackage{amsmath}
\usepackage{amsfonts}
\usepackage{amssymb}
\usepackage{graphicx}

\usepackage{listings}
\usepackage{xcolor}

\definecolor{mygreen}{rgb}{0,0.6,0}
\definecolor{mygray}{rgb}{0.5,0.5,0.5}
\definecolor{mymauve}{rgb}{0.58,0,0.82}

\lstdefinelanguage{JavaScript}{
  keywords={typeof, new, true, false, catch, function, return, null, catch, switch, var, if, in, while, do, else, case, break},
  keywordstyle=\color{blue}\bfseries,
  ndkeywords={class, export, boolean, throw, implements, import, this},
  ndkeywordstyle=\color{darkgray}\bfseries,
  identifierstyle=\color{black},
  sensitive=false,
  comment=[l]{//},
  morecomment=[s]{/*}{*/},
  commentstyle=\color{purple}\ttfamily,
  stringstyle=\color{red}\ttfamily,
  morestring=[b]',
  morestring=[b]",
}

\lstset{ %
  backgroundcolor=\color{white},   % choose the background color; you must add \usepackage{color} or \usepackage{xcolor}
  basicstyle=\small,        % the size of the fonts that are used for the code
  breakatwhitespace=false,         % sets if automatic breaks should only happen at whitespace
  breaklines=true,                 % sets automatic line breaking
  captionpos=b,                    % sets the caption-position to bottom
  commentstyle=\color{mygreen},    % comment style
  deletekeywords={...},            % if you want to delete keywords from the given language
  escapeinside={\%*}{*)},          % if you want to add LaTeX within your code
  extendedchars=true,              % lets you use non-ASCII characters; for 8-bits encodings only, does not work with UTF-8
  frame=single,	                   % adds a frame around the code
  keepspaces=true,                 % keeps spaces in text, useful for keeping indentation of code (possibly needs columns=flexible)
  keywordstyle=\color{blue},       % keyword style
  language=HTML,                 % the language of the code
  otherkeywords={*,...},           % if you want to add more keywords to the set
  numbers=left,                    % where to put the line-numbers; possible values are (none, left, right)
  numbersep=5pt,                   % how far the line-numbers are from the code
  numberstyle=\tiny\color{mygray}, % the style that is used for the line-numbers
  rulecolor=\color{black},         % if not set, the frame-color may be changed on line-breaks within not-black text (e.g. comments (green here))
  showspaces=false,                % show spaces everywhere adding particular underscores; it overrides 'showstringspaces'
  showstringspaces=false,          % underline spaces within strings only
  showtabs=false,                  % show tabs within strings adding particular underscores
  stepnumber=1,                    % the step between two line-numbers. If it's 1, each line will be numbered
  stringstyle=\color{mymauve},     % string literal style
  tabsize=2,	                   % sets default tabsize to 2 spaces
  title=\lstname,                   % show the filename of files included with \lstinputlisting; also try caption instead of title
  moredelim=**[is][\color{purple}]{@}{@},
}

\author{\\Universidade Federal de Goiás - UFG (Regional Jataí) \\Bacharelado em Ciência da Computação \\Física para Ciência da Computação \\Prof. Esdras Lins Bispo Jr.}

\title{
	{\sc \huge Lista de Exercícios 2} 
	\\{\tt Versão 2.1}
}

\begin{document}

\maketitle

\begin{enumerate}

\section{Conceitos}
	
	\item {\bf (Halliday 2.3)} Durante um espirro, os olhos podem se fechar por até 0,50 s. Se você está dirigindo um carro a 90 km/h e espirra, de quanto o carro pode se deslocar até você abrir novamente os olhos?
	
	\item {\bf (Halliday 2.5)} A posição de um objeto que se move ao longo de um eixo $x$ é dada por $x = 3t - 4t^2 + t^3$, em que $x$ está em metros e $t$ em segundos. Determine a posição do objeto para os seguintes valores de $t$: 
		\begin{enumerate}
			\item 1 s,
			\item 2 s,
			\item 3 s,
			\item 4 s,
			\item Qual é o deslocamento do objeto entre $t = 0$ s e $t = 4$ s?
			\item Qual é a velocidade média para o intervalo de tempo de $t = 2$ s a $t = 4$ s?
			\item Desenhe o gráfico de $x$ em função de t para $0 \leq t \leq 4$ s e indique como a resposta do item (f) pode ser
			determinada a partir do gráfico.
		\end{enumerate}
	
	\item {\bf (Halliday 2.14)} A função posição $x(t)$ de uma partícula que está se movendo ao longo do eixo $x$ é $x = 4,0 - 6,0t^2$, com $x$ em metros e $t$ em segundos. \label{q:2-14}
	\begin{enumerate}
		\item Em que instante e
		\item Em que posição a partícula para (momentaneamente)?
		\item Em que instante negativo e
		\item Em que instante positivo a partícula passa pela origem?
		\item Plote o gráfico de $x$ em função de $t$ \\
		para o intervalo de -5 s a + 5 s.
		\item Para deslocar a curva para a direita no gráfico,\\ devemos acrescentar a $x(t)$ o termo $+20t$ ou o termo $-20t$?
		\item Essa modificação aumenta ou diminui o valor de $x$ para o qual a partícula para momentaneamente?
	\end{enumerate}

\section{Programação}

	\item Em JavaScript, crie uma função {\tt velocidadeEscalarMedia} que receba quatro parâmetros: (i) {\tt x1} (posição inicial), (ii) {\tt x2} (posição final), (iii) {\tt t1} (instante inicial), e (iv) {\tt t2} (instante final). A função deve retornar um número (a velocidade escalar média). É necessário validar a entrada para garantir que  {\tt t2 - t1} seja positivo e não nulo. Se a entrada não for válida, a função deve imprimir, via {\tt console.log}, uma mensagem de erro.
	
	\item Em JavaScript, crie uma função {\tt posicao} que recebe {\tt t} como parâmetro (conforme equação apresentada na questão \ref{q:2-14}). A função deve retornar um número (a posição da partícula).
		
\end{enumerate}

\section{Referências}

\begin{itemize}
	\item HALLIDAY, D.; RESNICK, R.. Fundamentos de Física. Volume 1, Mecânica. 8ª Edição, LTC, Rio de Janeiro, 2011.

	\item RAMTAL, D.; DOBRE, A. Physics for JavaScript Games, Animation, and Simulations with HTML5 Canvas, Apress, 2014.
\end{itemize}

\end{document}